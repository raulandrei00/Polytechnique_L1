\documentclass{article}
\usepackage{graphicx} % Required for inserting images

\title{The influence of social media filter bubbles in public perception during elections}
\author{Raul Pop}


\begin{document}

\maketitle

\section{Abstract}

In this paper, I attempt to showcase how personalized social media algorithms can distort one's perception of the public opinion, in the crucial context of elections. This effect is the result of filter bubbles, that negatively impact the population in two ways:

1) A user that finds himself in a filter bubble with respect to the elections, may never come across helpful information from other points of view.

2) Someone affected by a filter bubble can believe that the support for a specific candidate or party is very strong and that they will surely win the election, hence losing interest in exercising their right to vote.

\section{Context}

The Romanian Presidential Elections were intended to take place on 2nd and 6th of December 2024, but have been canceled after the first round of voting due to alleged Russian interference in favor to a particular candidate: Calin Georgescu. He has been forbidden to candidate when the elections retook place in May 2025 but remained a relevant figure, as he enjoyed continuous support in spite of the accusations.
\\ The second tentative at Presidential elections saw two very distinct personalities as the main actors: George Simion and Nicusor Dan.

\subsection{George Simion}
George Simion advanced to the second round of voting with an astonishing 40.96\% of the total votes (3.862.761) in the first round. He is the founder of the AUR (Alliance for the Union of Romanians) far-right political party. This party was recently on the rise, as it was only founded in 2020 and is already the second largest in the Romanian parliament.
\\ He has studied Business and History in his youth and has since been an activist and campaigned for the unification of Romania and Moldova.
\\ His public presence is often considered uncivilized. As he has been aggresive on many occasions and had abusive outbursts towards women.
\\ George Simion has gained the support of many Calin Georgescu voters by saying he will name Georgescu prime minister.

\subsection{Nicusor Dan}
Nicusor Dan only advanced to the second round with 1.979.767 votes (20.99\%), less than one percent more than the third candidate.
\\ He ran as an independent candidate after being the mayor of Bucharest from 2020 to 2025.
\\ He completed his PhD studies at Ecole Normale Superieure and is a two time winner of the International Math Olympiad.
\\ His main focus during the campaign was keeping a pro-European stance and showcasing composure and critical thinking and reasoning.

\subsection{The campaign between the two voting rounds}

During the final week of campaign, \textbf{George Simion} greatly deteriorated his image through absenting from political debates hosted by national televisions, as well as making an appearance on french television, where he directed serious accusations at Emmanuel Macron, the president of France. This mainly sparked outrage in France as well as in Romania, but surprisingly, many people, french and romanians alike, praised him for his courage to speak up.

Meanwhile, \textbf{Nicusor Dan} visited many romanian communities, attended the debates where Simion did not show up and held an honorable public stance.

\subsection{The aftermath of the election}

Nicusor Dan won the second round of voting with 6.168.642 votes (53.6\%), meaning he tripled his vote count from the first tour. This is a record in Romania's six presidential elections that happened since 1989 when the communist regime was taken down. We also see significant growth in the number of exercised votes between the first and the second tour (approximately 22\%). This statistic is very similar to the 2014 presidential elections.

\section{Going forward}

Having presented the context and the results of this election, we can proceed to analyze the data as well as relevant samples of the content that circulated on social media to explore the phenomenon of filter bubbles created by the algorithm.
\\ By further studying the two candidates, it quickly becomes clear what personalities each of them represent: the demagogue and the composed, analytical agent.
\\ We can argue within reason that these two personalities are very different and, as such, social media algorithms will behave widely differently with people that have affinities to opposing sides.
\\ Since social media represents the main source of information for a significant demographic of the population, I claim that the nature of the algorithms it relies upon can segregate users into information bubbles that perpetually reinforce the same ideas. This can create high rigidity in the voters' opinions and options, hence making campaign efforts and political debates less relevant.

\section{Empirical evidence}
I will now provide and analyze an array of TikTok posts and social media phenomena, supporting either side of the campaign, that highlight my claim:

\subsection{Extremist interpretation of valid reasoning}

This$^{[1]}$ is a TikTok post by an account which has, suspiciously, exclusively posted content about the presidential elections since the annulled round in December 2024. It has seemingly personal posts since 2022, hence it is uncertain if it belongs to a real person or not. \\ The post in question shows an interview in which Nicusor Dan explains why the Social Democratic Party should continue being represented in the Parliament. 
It plays alarm sounds and exagerated warning messages, enticing users to rage towards this candidate. 
\\ This post seems to be intentionally crafted to induce these feelings, as rage and the promise of change have a very powerful effect on the demographic targeted by Simion's campaign: people with a narrow perspective that believe their insuccess and lack of opportunity is rooted in the system's inequity.
\\ From studying the comments section we can conclude that either the post successfully reached the target group, or that Nicusor Dan supporters did not bother to engage. Both cases are problematic since the latter further propagates the algorithmic information segregation.

\subsection{TikTok poll from a Simion-biased account}
\noindent
\begin{minipage}{0.3\textwidth}
  \includegraphics[width=\linewidth]{tiktok_poll.jpeg} % replace with your image file
\end{minipage}%
\hfill
\begin{minipage}{0.55\textwidth}
  This account clearly supported George Simion since it had his face as the profile picture, hence we can reason that it is more likely to show up on the "for you" page for someone that also supports him. This is confirmed by the engagement with the poll, where 96\% of the voters chose George Simion and only 4\% opted to vote for Nicusor Dan.
\end{minipage}

\subsection{My personal experience}

Before starting my empirical research for this paper on TikTok, I mainly used Instagram, where I would see quite frequent posts about the presidential election as it was approaching.
\\ I can say that every single post I have seen on Instagram has been in support of Nicusor Dan: from cool edits posted by "meme accounts" to genuine support from influential people. I personally found the edits of Nicusor Dan much more engaging than the ones I scouted on TikTok about Simion. In addition to that, I have not come across a single celebrity or person with a significant online presence that openly supported George Simion on any social media platform. 

\section{Interpretation of this evidence}

Based on my, as well as others', experience on social media during the elections and the samples I provided, we arrive to the conclusion that filter bubbles had an extreme effect of information segregation in the context of the 2025 Presidential Elections in Romania.

The opinion rigidity of the Simion-voters can be argumented using historical data as well as the events that transpired during the campaign.

\newpage

\section{References}

[1] https://vm.tiktok.com/ZNdrxNCkD/

\end{document}
